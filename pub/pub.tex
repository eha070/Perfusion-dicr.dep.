%% load necessary packages
\documentclass[paper=a4, fontsize=12pt,parskip=half,draft,headings=small]{scrartcl}
\usepackage{check-short}
\usepackage{overpic}


%% set author information etc
\title{Some title}
\author{Constantin Heck, Erlend Hodneland, Erik A. Hanson, \\ Arvid Lundervold, Jan Modersitzki, Alexandre Malyshev \\ (random order)}
\date{\today}


%% define some local commands
\newcommand{\Qso}{Q_{\mathrm{so}}}
\newcommand{\Qsi}{Q_{\mathrm{si}}}
\newcommand{\ca}{c_\mathrm{a}}
\newcommand{\CBV}{\mathrm{CBV}}

%define SI-units, since we want to be able to easily change them
\newcommand{\siq}{\cubic\milli\meter\per\second\per\square\milli\meter}
\newcommand{\siqt}{\cubic\milli\meter\per\second}
\newcommand{\siP}{\cubic\milli\meter\per\second\per\cubic\milli\meter}
\newcommand{\siQ}{\kilo\gram\per\second\per\cubic\milli\meter}
\newcommand{\sirho}{\kilo\gram\per\cubic\milli\meter}
\newcommand{\siJ}{\milli\mol\per\second\per\square\milli\meter}
\newcommand{\sic}{\milli\mol\per\cubic\milli\meter}
\newcommand{\siPn}{\milli\litre\per\minute\per100\milli\litre}

%--------------------------------------------------------------------------------------------
%--------------------------------------------------------------------------------------------
% Here begins the document, remove "draft" option in document class to include images
%--------------------------------------------------------------------------------------------
%--------------------------------------------------------------------------------------------
\begin{document}
	%--------------------------------------------------
	% Title Page=
	%--------------------------------------------------
	% \maketitle


	%--------------------------------------------------
	%--------------------------------------------------
	% Section: Mathematical theory
	%--------------------------------------------------
	%--------------------------------------------------
	\section{Mathematical Theory} \label{sec:maththeory}
		
	In order to model contrast agent (CA) propagation through the capillary system, several assumptions on flow and propagation have to be made.
	In pharmacokinetic analysis it is often assumed that each voxel is an autonomous system that can be modeled by standard pharmacokinetic theory [?].
	In fact, this assumption is questionable when we describe CA propagation through a larger area with a highly developed capillary system [?].
	Here we expect in contrast a coupled system, since each voxel can be regarded as an inlet for the surrounding voxels.
	We hence decided to describe the CA propagation as a spatially coupled transport process, which more accurately models the nature of the problem. 
	Details of the modeling of transport will be presented in Section \ref{sec:transport}.
	
	The CA transport itself is driven by blood flow. 
	We will hence describe a model for the blood flow in Section \ref{sec:flow}.
	Within the capillary system we expect the blood flow to be driven mainly by differences in pressure.
	This way of modeling flow comes from the theory of porous media and is expressed in Darcy's law [?].
	As a consequence, we will apply the physical understanding of flow as surface flow $q = q(x)$ with units \si{\siq}.
	The surface flow is a vector field describing the volume of fluid per unit time flowing across a sliced unit area of the sample.	
	However, a model to convert surface flow to perfusion with units \si{\siP} will be introduced in Section \ref{sec:flow2perf}.
	The resulting CA flux $J(x,t)$ with units \si{\siJ} is assumed to be linear and stationary, meaning that $J(x,t) = q(x)\cdot c(x,t)$.
	Apart from the normalization with respect to surface, the assumptions of linearity and stationarity are in complete agreement with standard pharmacokinetic modeling \cite{sourbron13}.
	A detailed description of modeling the blood flow can be found in Section \ref{sec:flow}.
	
	Another key ingredient from porous media is the introduction of the porosity $\phi$ for $0 \le \phi \le 1$.
	The porosity describes which fraction of a small tissue volume is accessible for blood.
	Comparing with pharmacokinetic modeling, the porosity directly translates to the cerebral blood volume (CBV).
	
	In order to test the standard models for their abilities to restore CBF and CBV, we need to convert the flow $q$ with units \si{\siq} to perfusion $P$ with units \si{\siP}. 
	A method to do that will be presented in Section \ref{sec:flow2perf}.

	In Section \ref{sec:CBV} we will give a proof that the porosity CBV can be estimated from the standard relationship $\CBV = \left(\int_0^\infty C(x,s) \diffint s\right)/\left(\int_0^\infty \ca(s) \diffint s\right)$.

	We will now describe the construction of the digital phantom in detail.
	
	
	%--------------------------------------------------
	% Subsection: Modeling the Blood Flow
	%--------------------------------------------------
	\subsection{Modeling the Blood Flow}\label{sec:flow}
	
	We model the blood flow as the flow of a fluid through a porous medium. 
	The fluid has units \si{\sirho} and is denoted by $\rho = \rho(x,t)$.
	As explained previously, the flow $q$ (in \si{\siq}) as well as the porosity $\phi$ (with $0 \le \phi \le 1$) are assumed to be stationary and hence independent of time.
	Fluid introduced and extracted from the system is modeled by a source- and sink term $\tilde{Q} = \tilde{Q}(x)$ with units \si{\siQ}. 
	The continuity equation describing conservation of mass now states that
	\begin{equation}
		\frac{\partial (\phi \rho)}{\partial t} + \nabla \cdot (\rho q) = \tilde{Q}.
		\label{eq:syntcont}
	\end{equation} 
	Furthermore assuming that the system is in steady-state and that the density of blood $\rho(x)$ is constant in space, we obtain that $\diffopcal{\rho}{t} = 0$ and hence:
	\begin{equation}
		\nabla \cdot q = \frac{\tilde{Q}}{\rho}.
		\label{eq:syntcontsimp}
	\end{equation}
	In order to scale away the density $\rho$ we define another source term $Q$ with units \si{\siQ} having the relation $\tilde{Q} = Q\rho$, thus transforming \eqref{eq:syntcontsimp} into
	\begin{equation}
		\nabla \cdot q = Q.
		\label{eq:syntcontsimp2}
	\end{equation}
	The right hand side is only non-zero within the source or the sink. 
	Elsewhere, \eqref{eq:syntcontsimp2} is concurrent with the incompressibility condition of divergence free flow.
	
	Low velocity fluid flow in porous media is described by Darcy's law [?]:
	\[
		q = -\frac{K}{\mu} \left( \nabla p + \rho g  \nabla z \right).
	\]
	here $g$ is the gravitational acceleration, $K$ is the permeability tensor with units \si{\square\milli\meter}, $z$ is the spatial position along the gravitational field and $\mu = \mu(x)$ is the viscosity of the fluid with units $\si{\pascal\second}$.
	For the current project the flow is taking place perpendicular to the gravitational field and the gravitational term can thereby be discarded.
	That a simplified version of Darcy's law thus becomes
	\begin{equation}
		q = -\frac{K}{\mu} \nabla p.
		\label{eq:syntdarcysimp}
	\end{equation}
	We now combine \eqref{eq:syntcontsimp2} and \eqref{eq:syntdarcysimp} and assume that $K$ is symmetric and positive definite.
	This yields the following elliptic partial differential equation the pressure-field $p$ must fulfill:
	\begin{equation}
		\left\vert
		\begin{alignedat}{2}
			\nabla \cdot \left( -\frac{K}{\mu} \nabla p \right) &= Q  \qquad &&x \in \Omega, \\
			n \cdot \nabla p &=0 &&x \in \partial \Omega.
		\end{alignedat}
		\right\vert
		\label{eq:flowmodel}
	\end{equation}
	Here $\partial \Omega$ denotes the boundary of $\Omega$ and $n(x)$ the outward unit normal vector. 
	Modeling an isolated system with no inflow or outflow along the boundary of $\Omega$, we impose the Neumann boundary conditions $n \cdot \nabla p = 0$ on $\partial \Omega$.
	Note that \eqref{eq:flowmodel} exhibits a solution which unique up to constant since only pressure differences are taken into consideration \cite{evans98}.
	Having solved \eqref{eq:flowmodel} the flow field can be computed according to \eqref{eq:syntdarcysimp} from the obtained pressure map. 
	
	
	%--------------------------------------------------
	% Subsection: Modeling the Contrast Agent Transport
	%--------------------------------------------------	
	\subsection{Modeling the Contrast Agent Transport}\label{sec:transport}
	In this section we will describe how the CA propagates in the tissue according to the flow-field.
	We assume that the CA is introduced at a source- and extracted at a sink location.
	The concentration map resulting from the simulation is later used to model the CA concentration one would observe with MRI or CT measurements.
	
	In order to define meaningful continuous contrast agent concentrations, we first describe the CA concentration in an (arbitrarily) small tissue volume $\Omega_i$.
	Assume that $V_i$ is the volume of $\Omega_i$ and $v_i$ the the blood volume of $\Omega_i$.
	By definition the porosity is in this case given by $\phi_i = v_i/V_i$.
	Letting first $V_i \rightarrow 0$ leads to the continuos porosity field $\phi(x)$.	
	We now fix a time point $t$.
	Denote the tracer concentration in $\Omega_i$ with respect to the whole volume $V_i$ by $C_i$ and the concentration with respect to the blood volume $v_i$ by $c_i$, both with units \si{\sic}. 
	From the definition of $c_i,C_i$ and $\phi_i$ we obtain the relation $C_i = \phi_i \cdot c_i$.
	Letting $V_i \to 0$ and assuming that all parameters are continuous allows us to introduce $C(x,t),c(x,t)$ as well as $\phi(x)$ at an arbitrary point.\todo{CH: Get the order continuous solution and continuous $C$ right}

	Assuming differentiable functions, the rate of change of tracer molecules within the control volume $\Omega_i$ can he described by
	\begin{equation}
		\diffop{t}\int_{\Omega_i}C \diffint x = \int_{\Omega_i}\diffop{t}(\phi c) \diffint x.
		\label{eq:dmdt}
	\end{equation}	
	Since we expect mainly transport along the vessels and marginal diffusion, the change in tracer mass within $\Omega_i$ occurs only from advective flow and the source and sink field $Q(x)$.
	Let us write the source- and the sink term as $Q(x) = \Qsi(x) + \Qso(x)$ where $\Qsi \le 0$ is the sink and $\Qso \ge 0$ is the source. 
	Both are assumed to be zero everywhere except at in the respective source and sink locations.
	Note that $Q = \Qso + \Qsi$ and $\int_\Omega Q \diffint x =0$ in line with incompressible flow. 
	The change in contrast agent at time point $t$ can hence be written as
	\begin{equation}
		-\int_{ \Gamma_i}c(q \cdot n)\diffint s + \int_{\Omega_i}\ca \Qso \diffint x + \int_{\Omega_i}c\Qsi \diffint x,
		\label{eq:surfflux}
	\end{equation}
	where $n$ is the outward unit normal on $\Gamma_i := \partial \Omega_i$.
	Furthermore $\ca = \ca(t)$ with unit \si{\sic} describes the amount of contrast agent entering the system at the source. 
	In standard pharmacokinetic modeling, $\ca$ is referred to as the arterial input function (AIF).
	From preservation of the tracer mass, equations \eqref{eq:dmdt} and \eqref{eq:surfflux} must balance such that
	\begin{equation}
		\int_{\Omega_i}\diffop{t} (\phi c) \diffint x + \int_{ \Gamma_i}c(q \cdot n) \diffint s = \int_{\Omega_i}\ca\Qso \diffint x + \int_{\Omega_i}c \Qsi \diffint x.
		\label{eq:conteq}
	\end{equation}
	Upon application of the divergence theorem, \eqref{eq:conteq} is consistent with the continuity equation on local form
	\begin{equation}
		\left\vert
		\begin{alignedat}{2}
			\phi \frac{\partial c}{\partial t} + \nabla \cdot (cq) &= \ca\Qso + c\Qsi \qquad	&x &\in \Omega, \ t>0,  \\
			c &= 0 																			 	&x &\in \Omega, \ t=0.
		\end{alignedat}
		\right\vert
		\label{eq:conteqlocal}
	\end{equation}
	This is a linear transport equation in $c(x,t)$. 
	Assuming that $\phi$ is Lipschitz continuous and that $\Qso,\Qsi,\ca$ are continuous, we can follow that $q$ is as the solution of \eqref{eq:flowmodel} also Lipschitz continuous.
	In this case we can follow \cite{evans98} to see that \eqref{eq:conteqlocal} exhibits a unique local solution.
	
	
	%--------------------------------------------------
	% Subsection: Converting Flow to Perfusion
	%--------------------------------------------------
	\subsection{Converting Flow to Perfusion}\label{sec:flow2perf}
	The model described in \eqref{eq:flowmodel} uniquely determines the flux field $q(x)$. 
	However, in pharmacokinetic modeling the parameter of interest is usually the cerebral blood flow (CBF), which we will denote by $P(x)$.
	It is not obvious how to transform a flux field into a scalar perfusion field $P(x)$.
	There are at least two obvious differences between $q$ and $P$. 
	First, the flux is a vector field and the perfusion is a scalar field. 
	Second, the flux relates to a surface area and the perfusion relates to a volume. 
	Thus, these to quantities are strictly, mathematically different but still conceptually related. 
	In the following we describe a method for converting flow into perfusion.

	The classical understanding of perfusion is the amount of blood feeding a tissue volume per unit time. 
	Thus, the perfusion $P$ has the units \si{\siP}.
	It is common to scale this quantity to normalized perfusion $P_n(x)$ with units \si{\siPn}. 
	One approach for converting flux into perfusion could be to estimate the perfusion as the total inflow (or outflow) of fluid (e.g. arterial blood) into a control region per unit time, and then normalizing with the control region volume. 
	This is only a valid approach if every control region is separated from other control regions, and not feeding each other. 
	Thus, this approach is valid for an entire organ being the control region. 
	Such understanding is in line with the understanding of classical compartment models for perfusion where each voxel has its own source of feeding arterial blood, independent of the neighbor voxels. 
	Clearly, this is a simplification since voxels will be fed by their neighbors. 
	In our synthetic flow model SFM as well as in normal tissue this assumption is violated since the voxels are feeding their neighbor voxels with arterial blood. 
	Simply summing the total inflow into a voxel and dividing by the voxel volume will strongly over-estimate the perfusion since we would divide by the wrong normalization volume. 
	The problematic issue is that the incoming blood is feeding more voxels than the current voxel. 
	This phenomenon is demonstrated in Fig. \ref{fig:perfusion-problem} where the volume on the left has the true perfusion value of $P_1 = F_0/(2V)$ for an incoming flow $F_0$ in \si{\siqt} and distribution volume $2V$ in \si{\cubic\milli\meter}. 
	However, for another discretization as shown in the middle, the perfusion within each of these sub-volumes becomes $P_2 = F_0/V = 2P_1$. 
	Taking the average across both sub-volumes, it is clear that the perfusion is over-estimated with a factor of two. 
	A discretization dependent perfusion value is not desirable, and the perfusion estimate of $P_2$ is clearly wrong. 


	\begin{figure}[H]
	    \centering
	    \begin{overpic}[scale=0.5]{figs/perfusion-problem.eps}
	    	\put(11,67){\color{black}$F_0$}
			\put(49,67){\color{black}$F_0$}
			\put(85.2,66){\color{black}$\Delta F_0$}
			\put(13,33){\color{black}$2V$}
			\put(50,20){\color{black}$V$}
			\put(50,45){\color{black}$V$}
			\put(91,42){\color{black}$\Delta V$}
		\end{overpic}
	    \caption{Perfusion within a small volume. Left: A compartment with volume $2V$ is exposed to a flow $F_0$ $\si{\siqt}$ of fluid. From definition, the overall perfusion within this object becomes $P_1 = F_0/(2V)$ \si{\siq}. Right: The volume is divided into two smaller compartment (e.g. voxels), and the perfusion for each of the compartments becomes $P_2 = F_0/V = 2P_1$. This discrepancy between the two discretisation regimes occurs because the flow is counted twice as it is feeded from one voxel to the other. Right: As a solution to the described problem we pick out a true distribution volume $\Delta V$ (area in this 2D sketch), which is a small area around a given streamline along the centre line of the grey area. This is the true distribution volume (area) which is feeded with arterial blood from the incoming fractional flow $\Delta F_0$. The correct perfusion within $\Delta V$ is therefore $\Delta F_0/dV$. The entire compartment can further be divided into infinitesimal distribution volumes, thus providing voxelwise perfusion values.}
	    \label{fig:perfusion-problem}
	\end{figure}

	The reason for this discrepancy is that for $P_2$ the perfusion has been counted twice since we are dividing by the wrong distribution volume. 
	Instead, we need to consider to the classical definition of perfusion. 
	The concept of perfusion has a very precise meaning, as the amount of arterial blood per time unit delivered to a capillary bed in a biological tissue, and then scaled by the feeded tissue volume. 
	Therefore, we must divide the incoming flow by the total distribution volume that is covered by the fluid streamlines. 
	This formulation coincides with the classical understanding of perfusion, and the correct distribution volume will rather be the volume that the fluid particles within an infinitesimal cross-sectional area around the streamlines are covering. 
	Assuming laminar flow, the streamlines are not crossing each other and we can estimate the true distribution volume that is fed by a given arterial blood flow.

	Let us fix a point $y \in \Omega$.
	Formally, the direction of the streamline passing $y$ is identical to the flow $q(y)$ at this location. 
	Let $A_\varepsilon$ be a 2-D disc with radius $\varepsilon$ which is orthogonal to $q(y)$.
	The total flow of particles \si{\siqt} over the disc can be expressed as
	\[
		F = \int_{A_\varepsilon} (q \cdot n) \diffint A = \int_{A_\varepsilon} \left(q\cdot \frac{q(y)}{\Vert q(y) \Vert}\right) \diffint A.
	\]
	Let $l \subset \Omega$ be the (unique) streamline passing through $y$ and let $L$ denote be the length of $l$.
	We now consider a small tube $T$ of radius $\varepsilon$ around $l$.
	Following Cavalieri's principle the volume $V$ of $T$ can be expressed as $V = \varepsilon^2\pi L$
	Hence we can express the perfusion $P_\varepsilon$ for this volume as
	\[
		P_\varepsilon = \frac{F}{V} = \frac{1}{\varepsilon^2 \pi L} \int_{A_\varepsilon} \left( q\cdot \frac{q(y)}{\Vert q(y) \Vert} \right) \diffint A. 
	\]
	Letting $\varepsilon \to 0$ and using that $F/(\varepsilon^2\pi)$ goes to $\Vert q(y) \Vert$ if $q(x)$ is continuous yields a value for the perfusion $P_{y}$:
	\begin{equation}
		P_{y} = \frac{\Vert q(y) \Vert}{L}
		\label{eq:flow2perf}
	\end{equation}
	where $L$ is the length of the streamline passing through the point $y$. 
	This is an explicit formula for converting flux into perfusion and is later used for evaluation of the classical model for perfusion. 


	%--------------------------------------------------
	% Subsection: Estimate the Porosity
	%--------------------------------------------------	
	\subsection{A Method to Estimate the Porosity}\label{sec:CBV}
	
	
	The principle of mass balance of fluid and tracer particles is described by \eqref{eq:flowmodel} and \eqref{eq:conteqlocal}. 
	For locations $x$ where $Q(x) = 0$ one can see that \eqref{eq:conteqlocal} becomes
	\[
		\phi\frac{\partial c}{\partial t}  = - q \cdot \nabla c.
		\label{eq:1cmodel}
	\]
	Integrating from $t_0$ to $t_1$ results in the model
	\[
		\phi [c(x,t_1) - c(x,t_0)]  = - \int_{t_0}^{t_1}q \cdot  \nabla c \diffint t.
	\]
	Going to the limit $t_0 = 0$ to $t_1 = \infty$, using the boundary conditions $c(x,0) = c(x,\infty) = 0$ and defining $E(x):= \int_0^\infty c(x,s) \diffint s$ leads to
	\[
		0 = q \cdot \nabla  E(x).
		\label{eq:streamlinezero}
	\]
	We can interpret this equation in such a way, that $q$ lies in-plane with the level-sets of the function $E(x)$.
	This means that $E(x)$ is constant along the streamlines of the fluid flow.
	Since we assumed that $Q$ has a delta-like structure and since all streamlines are emerging at the arterial-input, we obtain
	\[
		\int_0^\infty c(x,s) \diffint s = \int_0^\infty \ca(s) \diffint s.
	\]
	Using $C(x,t) = \phi(x) c(x,t)$ we obtain
	\begin{equation}
		\phi(x) =  \frac{ \int_{0}^{\infty} C(x,s) \diffint s }{ \int_{0}^{\infty} \ca(s) \diffint s}.
		\label{eq:phi}
	\end{equation}
	This expression coincides with the classical formula for CBV and is hereby proven analytically for the proposed model.
	
	
	
	
	 
	
	
	
	
	
	


	%--------------------------------------------------
	% Bibliography
	%--------------------------------------------------
	\bibliographystyle{ieeetr}	
	\bibliography{./bibliography.bib}

	
\end{document}