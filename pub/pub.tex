%% load necessary packages
\documentclass[paper=a4, fontsize=12pt,parskip=half,draft,headings=small]{scrartcl}
\usepackage{check-short}

%% set author information etc
\title{Some title}
\author{Constantin Heck, Erlend Hodneland, Erik A. Hanson, \\ Arvid Lundervold, Jan Modersitzki, Alexandre Malyshev \\ (random order)}
\date{\today}


%% define some local commands
\newcommand{\Qso}{Q_{\mathrm{so}}}
\newcommand{\Qsi}{Q_{\mathrm{si}}}
\newcommand{\ca}{c_\mathrm{a}}


%--------------------------------------------------------------------------------------------
%--------------------------------------------------------------------------------------------
% Here begins the document
%--------------------------------------------------------------------------------------------
%--------------------------------------------------------------------------------------------
\begin{document}
	%--------------------------------------------------
	% Title Page=
	%--------------------------------------------------
	% \maketitle


	%--------------------------------------------------
	%--------------------------------------------------
	% Section: Mathematical theory
	%--------------------------------------------------
	%--------------------------------------------------
	\section{Mathematical Theory} \label{sec:maththeory}
	
	In order to model contrast agent (CA) propagation through the capillary system, several assumptions on flow and propagation have to be made.
	In pharmacokinetic analysis it is often assumed that each voxel is an autonomous system that can be modeled by standard pharmacokinetic theory [?].
	In fact, this assumption is questionable when we describe CA propagation through a larger area with a highly developed capillary system [?].
	Here we expect in contrast a coupled system, since each voxel can be regarded as an inlet for the surrounding voxels.
	We hence decided to describe the CA propagation as a spatially coupled transport process, which more accurately models the nature of the problem. 
	Details of the modeling of transport will be presented in Section \ref{sec:transport}.
	
	The CA transport itself is driven by blood flow. 
	We will hence describe a model for the blood flow in Section \ref{sec:flow}.
	Within the capillary system we expect the blood flow to be driven mainly by differences in pressure.
	This way of modeling flow comes from the theory of porous media and is expressed in Darcy's law [?].
	As a consequence, we will apply the physical understanding of flow as surface flow $q = q(x)$ with units \si{\cubic\cubic\milli\meter\per\second\per\square\milli\meter}.
	The surface flow is a vector field describing the volume of fluid per unit time flowing across a sliced unit area of the sample.	
	However, a model to convert surface flow to perfusion with units \si{\cubic\cubic\milli\meter\per\second\per\cubic\milli\meter} will be introduced in Section \ref{sec:flow2perf}.
	The resulting CA flux $J(x,t)$ with units \si{\milli\mol\per\second\per\square\milli\meter} is assumed to be linear and stationary, meaning that $J(x,t) = q(x)\cdot c(x,t)$.
	Apart from the normalization with respect to surface, the assumptions of linearity and stationarity are in complete agreement with standard pharmacokinetic modeling \cite{sourbron13}.
	A detailed description of modeling the blood flow can be found in Section \ref{sec:flow}.
	
	Another key ingredient from porous media is the introduction of the porosity $\phi$ where $0 \le \phi \le 1$.
	The porosity describes which fraction of a small tissue volume is accessible for blood.
	Comparing with pharmacokinetic modeling, the porosity directly translates to the cerebral blood volume (CBV).

	We will now describe the construction of the digital phantom in detail.
	
	
	%--------------------------------------------------
	% Subsection: Modeling the Blood Flow
	%--------------------------------------------------
	\subsection{Modeling the Blood Flow}\label{sec:flow}
	
	We model the blood flow as the flow of a fluid through a porous medium. 
	The fluid has units \si{\kilo\gram\per\cubic\milli\meter} and is denoted by $\rho = \rho(x,t)$.
	As explained previously, the flow $q$ (in \si{\cubic\milli\meter\per\second\per\square\milli\meter}) as well as the porosity $\phi$ (with $0 \le \phi \le 1$) are assumed to be stationary and hence independent of time.
	Fluid introduced and extracted from the system is modeled by a source- and sink term $\tilde{Q} = \tilde{Q}(x)$ with units \si{\kilo\gram\per\second\per\cubic\milli\meter}. 
	The continuity equation describing conservation of mass now states that
	\begin{equation}
		\frac{\partial (\phi \rho)}{\partial t} + \nabla \cdot (\rho q) = \tilde{Q}.
		\label{eq:syntcont}
	\end{equation} 
	Furthermore assuming that the system is in steady-state and that the density of blood $\rho(x)$ is constant in space, we obtain that $\diffopcal{\rho}{t} = 0$ and hence:
	\begin{equation}
		\nabla \cdot q = \frac{\tilde{Q}}{\rho}.
		\label{eq:syntcontsimp}
	\end{equation}
	In order to scale away the density $\rho$ we define another source term $Q$ with units \si{\cubic\milli\meter\per\second\per\cubic\milli\meter} having the relation $\tilde{Q} = Q\rho$, thus transforming \eqref{eq:syntcontsimp} into
	\begin{equation}
		\nabla \cdot q = Q.
		\label{eq:syntcontsimp2}
	\end{equation}
	The right hand side is only non-zero within the source or the sink. 
	Elsewhere, \eqref{eq:syntcontsimp2} is concurrent with the incompressibility condition of divergence free flow.
	
	Low velocity fluid flow in porous media is described by Darcy's law [?]:
	\[
		q = -\frac{K}{\mu} \left( \nabla p + \rho g  \nabla z \right).
	\]
	here $g$ is the gravitational acceleration, $K$ is the permeability tensor with units \si{\square\milli\meter}, $z$ is the spatial position along the gravitational field and $\mu = \mu(x)$ is the viscosity of the fluid with units $\si{\pascal\second}$.
	For the current project the flow is taking place perpendicular to the gravitational field and the gravitational term can thereby be discarded.
	That a simplified version of Darcy's law thus becomes
	\begin{equation}
		q = -\frac{K}{\mu} \nabla p.
		\label{eq:syntdarcysimp}
	\end{equation}
	We now combine \eqref{eq:syntcontsimp2} and \eqref{eq:syntdarcysimp} and assume that $K$ is symmetric and positive definite.
	This yields the following elliptic partial differential equation the pressure-field $p$ must fulfill:
	\begin{equation}
		\left\vert
		\begin{alignedat}{2}
			\nabla \cdot \left( -\frac{K}{\mu} \nabla p \right) &= Q  \qquad &&x \in \Omega, \\
			n \cdot \nabla p &=0 &&x \in \partial \Omega.
		\end{alignedat}
		\right\vert
		\label{eq:flowmodel}
	\end{equation}
	Here $\partial \Omega$ denotes the boundary of $\Omega$ and $n(x)$ the outward unit normal vector. 
	Modeling an isolated system with no inflow or outflow along the boundary of $\Omega$, we impose the Neumann boundary conditions $n \cdot \nabla p = 0$ on $\partial \Omega$.
	Note that \eqref{eq:flowmodel} exhibits a solution which unique up to constant since only pressure differences are taken into consideration \cite{evans1998}.
	Having solved \eqref{eq:flowmodel} the flow field can be computed according to \eqref{eq:syntdarcysimp} from the obtained pressure map. 
	
	
	%--------------------------------------------------
	% Subsection: Modeling the Contrast Agent Transport
	%--------------------------------------------------	
	\section{Modeling the Contrast Agent Transport}\label{sec:transport}
	In this section we will describe how the CA propagates in the tissue according to the flow-field.
	We assume that the CA is introduced at a source- and extracted at a sink location.
	The concentration map resulting from the simulation is later used to model the CA concentration one would observe with MRI or CT measurements.
	
	In order to define meaningful continuous contrast agent concentrations, we first describe the CA concentration in an (arbitrarily) small tissue volume $\Omega_i$.
	Assume that $V_i$ is the volume of $\Omega_i$ and $v_i$ the the blood volume of $\Omega_i$.
	By definition the porosity is in this case given by $\phi_i = v_i/V_i$.
	Letting first $V_i \rightarrow 0$ leads to the continuos porosity field $\phi(x)$.	
	We now fix a time point $t$.
	Denote the tracer concentration in $\Omega_i$ with respect to the whole volume $V_i$ by $C_i$ and the concentration with respect to the blood volume $v_i$ by $c_i$, both with units \si{\milli\mol\per\cubic\milli\meter}. 
	From the definition of $c_i,C_i$ and $\phi_i$ we obtain the relation $C_i = \phi_i \cdot c_i$.
	Letting $V_i \to 0$ and assuming that all parameters are continuous allows us to introduce $C(x,t),c(x,t)$ as well as $\phi(x)$ at an arbitrary point.

	Assuming differentiable functions, the rate of change of tracer molecules within the control volume $\Omega_i$ can he described by
	\begin{equation}
		\diffop{t}\int_{\Omega_i}C \diffint x = \int_{\Omega_i}\diffop{t}(\phi c) \diffint x.
		\label{eq:dmdt}
	\end{equation}	
	Since we expect mainly transport along the vessels and marginal diffusion, the change in tracer mass within $\Omega_i$ occurs only from advective flow and the source and sink field $Q(x)$.
	Let us write the source- and the sink term as $Q(x) = \Qsi(x) + \Qso(x)$ where $\Qsi \le 0$ is the sink and $\Qso \ge 0$ is the source. 
	Both are assumed to be zero everywhere except at in the respective source and sink locations.
	Note that $Q = \Qso + \Qsi$ and $\int_\Omega Q \diffint x =0$ in line with incompressible flow. 
	The change in contrast agent at time point $t$ can hence be written as
	\begin{equation}
		-\int_{ \Gamma_i}c(q \cdot n)\diffint s + \int_{\Omega_i}\ca \Qso \diffint x + \int_{\Omega_i}c\Qsi \diffint x,
		\label{eq:surfflux}
	\end{equation}
	where $n$ is the outward unit normal on $\Gamma_i := \partial \Omega_i$.
	Furthermore $\ca = \ca(t)$ with unit \si{\milli\mol\per\cubic\milli\meter} describes the amount of contrast agent entering the system at the source. 
	In standard pharmacokinetic modeling, $\ca$ is referred to as the arterial input function (AIF).
	From preservation of the tracer mass, equations \eqref{eq:dmdt} and \eqref{eq:surfflux} must balance such that
	\begin{equation}
		\int_{\Omega_i}\diffop{t} (\phi c) \diffint x + \int_{ \Gamma_i}c(q \cdot n) \diffint s = \int_{\Omega_i}\ca\Qso \diffint x + \int_{\Omega_i}c \Qsi \diffint x.
		\label{eq:conteq}
	\end{equation}
	Upon application of the divergence theorem, \eqref{eq:conteq} is consistent with the continuity equation on local form
	\begin{equation}
		\phi \frac{\partial c}{\partial t} + \nabla \cdot (cq) = \ca\Qso + c\Qsi,
		\label{eq:conteqlocal}
	\end{equation}
	a linear transport equation in $c(x,t)$. 
	
	
	
	
	\section{Converting Flow to Perfusion}\label{sec:flow2perf}
	
	
	
	
	
	
	
	 
	
	
	
	
	
	


	%--------------------------------------------------
	% Bibliography
	%--------------------------------------------------
	\bibliographystyle{ieeetr}	
	\bibliography{./bibliography.bib}

	
\end{document}